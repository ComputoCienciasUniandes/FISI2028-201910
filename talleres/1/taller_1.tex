
%--------------------------------------------------------------------
%--------------------------------------------------------------------
% Formato para los talleres del curso de Métodos Computacionales
% Universidad de los Andes
%--------------------------------------------------------------------
%--------------------------------------------------------------------

\documentclass[11pt,letterpaper]{exam}
\usepackage[utf8]{inputenc}
\usepackage[spanish]{babel}
\usepackage{graphicx}
\usepackage{tabularx}
\usepackage[absolute]{textpos} % Para poner una imagen en posiciones arbitrarias
\usepackage{multirow}
\usepackage{float}
\usepackage{hyperref}
\decimalpoint

\begin{document}
\begin{center}
{\Large Métodos Computacionales} \\
Taller 1 --- 2019-10\\

\end{center}

%\begin{textblock*}{40mm}(10mm,20mm)
%  \includegraphics[width=3cm]{logoUniandes.png}
%\end{textblock*}

%\begin{textblock*}{40mm}(164mm,20mm)
%  \includegraphics[width=3cm]{logoUniandes.png}
%\end{textblock*}

\vspace{0.3cm}

\noindent
La solución a este taller debe subirse por SICUA antes de las 11:00PM
del lunes 11 de Marzo.
La soluci\'on debe estar dentro de un archivo
\verb"NombreApellido_taller1.py".
Por ejemplo, si su nombre es Miranda
July deber\'ia subir el archivo \verb"MirandaJuly_taller1.py" y
adentro  debe encontrarse el c\'odigo de Python con las soluciones. 
Las funciones propuestas {\bf {deben}} tener el nombre y los
par\'ametros pedidos para poder ser calificadas. 
Las funciones ser\'an calificadas a partir de entradas definidas por
el evaluador. 100 puntos corresponden a una nota de $5.0$.
{\bf No se permite reutilizaci\'on de c\'odigos escritos por terceros (esto
incluye repositorios asociados al curso).}

\vspace{0.3cm}
\begin{questions}




\question{{\bf Equilibrio} (20 puntos)}

El objetivo es encontrar el \'angulo $\theta$ en la Figura
\ref{fig:uno} cuando el sistema se encuentra en equilibrio.  

La funci\'on de Python que resuelve el problema debe definirse como:

\verb"def equilibrio(m, q, l, g):"

donde:
\begin{itemize}
\item\verb"m" es la masa en gramos de las masas.
\item\verb"q" es la carga en nanocoulomb de las masas.
\item\verb"l" es la longitud en metros de las cuerdas.
\item\verb"g" es la aceleraci\'on de la gravedad en m/s$^2$.
\end{itemize}

La funci\'on debe terminar con un \verb"return theta", donde
\verb"theta" es el \'angulo de equilibrio en radianes.
La funci\'on debe utilizar el m\'etodo de Newton-Rhapson para
encontrar el \'angulo \verb"theta".


\begin{figure}[h]
\begin{center}
\includegraphics[width=5cm]{bolas.jpeg}
\end{center}
\caption{Esquema para el Ejercicio 1.}
\label{fig:uno}
\end{figure}


\question {{\bf Ajuste Polinomial M\'inimos Cuadrados Matricial} (30 puntos)} 


El objetivo de este ejercicio es hacer el ajuste de un conjunto de $n$
puntos $x_i$, $y_i\pm \sigma_i$ a un modelo polinomial de grado $N$:

\begin{equation}
y = \sum_{j=0}^{N}a_j x^{j}.
\label{eq:modelo}
\end{equation}

La funci\'on de Python que resuelve el problema debe definirse como 


\verb"def matrix_polynomial(filename, poly_degree=2):".  

donde:

\begin{itemize}
\item \verb"filename" corresponde al nombre del archivo de texto con
  tres columnas con los valores de $x_i$, $y_i$, $\sigma_i$.
\item \verb"poly_degree" es el grado del polinomio ($N$ en
  la Ecuaci\'on \ref{eq:modelo}).
\end{itemize}

La funci\'on debe terminar con un \verb"return a", donde \verb"a" es un
array de numpy con los valores de los coeficientes del polinomio de
manera que \verb"a[j]" corresponda a $a_j$. El array \verb"a" debe
tener entonces una longitud igual a \verb"N+1".  

El ajuste debe hacerse con una formulaci\'on matricial de m\'inimos
cuadrados y resoluci\'on de sistema de ecuaciones por el m\'etodo de
eliminaci\'on Gaussiana.
No se permite utilizar funciones de \'algebra lineal de librerias.


\question {{\bf Ajuste Polinomial Markov Chain Monte Carlo} (50 puntos)} 

El objetivo es el mismo que en el numeral anterior pero con un
m\'etodo Metr\'opolis-Hastings.

La funci\'on de Python que resuelve el problema debe definirse como 

\verb"def MCMC_polynomial(filename, poly_degree=2, n_steps=50000):".  


donde:

\begin{itemize}
\item \verb"filename" corresponde al nombre de archivo de texto con
  tres columnas con los valores de $x_i$, $y_i$, $\sigma_i$.
\item \verb"poly_degree" es el grado del polinomio ($N$ en
  la Ecuaci\'on \ref{eq:modelo}).
\item \verb"n_steps" es el n\'umero de pasos de las cadenas de Markov.
\end{itemize}

La funci\'on debe terminar con un \verb"return a" con las mismas
condiciones descritas para \verb"a" en el caso la funci\'on
\verb"matrix_polynomial". 

\end{questions}






\end{document}
