\documentclass[letterpaper,10pt,onecolumn]{article}
\usepackage[spanish]{babel}
\usepackage[utf8]{inputenc}
\usepackage{amsfonts}
\usepackage{amsthm}
\usepackage{amsmath}
\usepackage{mathrsfs}

\usepackage{enumitem}
\usepackage[pdftex]{color,graphicx}
\usepackage{hyperref}
\usepackage{listings}
\usepackage{calligra}
\usepackage{url}
%\usepackage{algpseudocode} 
\DeclareMathAlphabet{\mathcalligra}{T1}{calligra}{m}{n}
\DeclareFontShape{T1}{calligra}{m}{n}{<->s*[2.2]callig15}{}
\newcommand{\scripty}[1]{\ensuremath{\mathcalligra{#1}}}
\lstloadlanguages{[5.2]Mathematica}
\setlength{\oddsidemargin}{0cm}
\setlength{\textwidth}{490pt}
\setlength{\topmargin}{-40pt}
\addtolength{\hoffset}{-0.3cm}
\addtolength{\textheight}{4cm}

\begin{document}
\begin{center}

\includegraphics[width=490pt]{header.png}\\[0.5cm]

\textsc{\LARGE M\'etodos Computacionales}\\[0.1cm]
%\large Jaime E. Forero Romero\\[0.5cm]

\end{center}

\large \noindent\textsc{Nombre del curso:}  M\'etodos Computacionales%Aqui  
                                %nombre del curso 
  
\noindent\textsc{C\'odigo del curso:} FISI 2028 (Magistral) FISI 2029 (Laboratorio) %Aqui el codigo del
                                %curso 

\noindent\textsc{Unidad acad\'emica:} Departamento de F\'isica

\noindent\textsc{Periodo acad\'emico:} 201910 %Aqui el periodo,
                                %p.ej. 201510 

\noindent\textsc{Horario (magistral - Secci\'on 1):} Ma y Ju, 6:30 a 7:50.

\noindent\textsc{Horario (magistral - Secci\'on 2):} Ma y Vie, 17:00 a 18:20.
%Aqui el horario, %p.ej. Ma y Ju, 10:00 a 11:20 

\noindent\textsc{Horario (laboratorio - Secci\'on 1):} Ma 15:30 - 16:50

\noindent\textsc{Horario (laboratorio - Secci\'on 2):} Mie 14:00 - 15:20

\noindent\rule{\textwidth}{1pt}\\[-0.3cm]

\normalsize \noindent\textsc{Nombre profesor magistral (Secci\'on 1):}
Jaime E. Forero Romero%Aqui nombre del profesor principal   

\noindent\textsc{Correo electr\'onico:}
\href{mailto:je.forero@uniandes.edu.co}{\nolinkurl{je.forero@uniandes.edu.co}}
%Cambie address por su direccion de correo uniandes 

\noindent\textsc{Horario y lugar de atenci\'on:} Ma y , 15:00 a 17:00, Oficina Ip208 
\\[-0.1cm]

\normalsize \noindent\textsc{Nombre profesor magistral (Secci\'on 2):}
Veronica Arias Callejas %Aqui nombre del profesor principal 

\noindent\textsc{Correo electr\'onico:}
\href{mailto:v.arias@uniandes.edu.co}{\nolinkurl{v.arias@uniandes.edu.co}}
%Cambie address por su direccion de correo uniandes 

\noindent\textsc{Horario de atenci\'on:} con cita previa. 
\\[-0.1cm]

\noindent\textsc{Nombre profesor(a) complementario(a):} 
%Jesus David Prada Gonzalez
%Aqui nombre
                                %del profesor complementario si aplica 

\noindent\textsc{Correo electr\'onico:}
%\href{mailto: jd.prada1760@uniandes.edu.co}{\nolinkurl{jd.prada1760@uniandes.edu.co}}

%Cambie address por direccion de correo uniandes del profesor
%complementario 

%\noindent\textsc{Horario y lugar de atenci\'on:} %Aqui horario y
%lugar de atencion del profesor complementario, p.ej. Vi, 15:00 a
%17:00, Oficina Ip102 
%\\[-0.1cm]
%Repetir esto en caso de varios profesores complementarios

%\noindent\textsc{Nombre monitor(a):} %Aqui nombre del monitor si aplica

%\noindent\textsc{Correo electr\'onico:}
%\href{mailto:address@uniandes.edu.co}{\nolinkurl{address@uniandes.edu.co}}
%%Cambie address por direccion de correo uniandes del monitor 

%\noindent\textsc{Horario y lugar de atenci\'on:} %Aqui horario y
%lugar de atencion del monitor, p.ej. Vi, 15:00 a 17:00, Oficina Ip102 

\noindent\rule{\textwidth}{1pt}\\[-0.1cm]

\newcounter{mysection}
\addtocounter{mysection}{1}

\noindent\textbf{\large \Roman{mysection} \quad Introducci\'on}\\[-0.2cm]

%Este espacio es para hacer una introduccion al curso, evidenciando la
%propuesta metodologica. Debe ser clara y precisa. 

\noindent\normalsize Los m\'etodos computacionales son un aspecto
inseparable de cualquier \'area de trabajo en ciencia e ingenier\'ia.
Esto se debe a la facilidad de acceso a 
computadoras programables  y al aumento exponencial en su capacidad de
procesamiento. 
Este curso busca guiar a los estudiantes en el desarrollo de sus
\emph{habilidades computacionales} en generar y procesar
datos para obtener informaci\'on sobre la realidad que esos datos
pretenden describir.  
Estos datos pueden ser mediciones o simulaciones de
sistemas f\'isicos, qu\'imicos, geol\'ogicos, biol\'ogicos,
financieros o industriales, entre otros.     
El programa del curso tiene dos componentes diferenciadas: m\'etodos
num\'ericos  y \emph{carpinter\'ia} de software.  
La parte de m\'etodos num\'ericos busca ilustrar el paso entre la
formulaci\'on matem\'atica de una pregunta sobre la realidad y su
descripci\'on num\'erica/computacional.
El objetivo principal es mostrar las diferentes formas de usar el
software para escribir esa pregunta.
La parte de carpinter\'ia busca presentar algunas t\'ecnicas y
pr\'acticas necesarias, no suficientes, para poder obtener resultados
computacionales reproducibles.    
\\[0.1cm] 

\stepcounter{mysection}
\noindent\textbf{\large \Roman{mysection} \quad Objetivos}\\[-0.2cm]

%En este espacio se debe precisar el ente visor del curso y el
%proposito ideal al finalizar el curso. 
\noindent\normalsize El objetivo principal del curso es presentar
algoritmos y t\'ecnicas b\'asicas para:

\begin{itemize}
\item resolver ecuaciones diferenciales ordinarias y parciales, \\[-0.6cm]
\item analizar y describir datos con t\'ecnicas estad\'isticas basadas
  en m\'etodos Monte Carlo, \\[-0.6cm]
\item desarrollar esquemas reproducibles para el an\'alisis de datos cient\'ificos. \\[-0.6cm]
\end{itemize} 
\vspace*{0.5cm} 

\stepcounter{mysection}
\noindent\textbf{\large \Roman{mysection} \quad Competencias a
  desarrollar}\\[-0.2cm] 

%En este espacio se describen las habilidades que el estudiante desarrollara en el transcurso del curso.

\noindent\normalsize Al finalizar el curso, se espera que el
estudiante est\'e en capacidad de: 

\begin{itemize}
\item implementar algoritmos sencillos para la resoluci\'on de
  ecuaciones diferenciales y para el an\'alisis estad\'istico
  exploratorio de datos, \\[-0.6cm]   
\item manejar la sintaxis b\'asica de dos lenguajes de programaci\'on
  modernos de computaci\'on num\'erica: uno de alto nivel y otro de
  bajo nivel,\\[-0.6cm]  
\item desarrollar un esquema sencillo para obtener resultados
  computacionales reproducibles.\\[-0.6cm]  
\end{itemize}

\vspace*{0.5cm} 

\stepcounter{mysection}
\noindent\textbf{\large \Roman{mysection} \quad Contenido por
  semanas}\\[-0.2cm]  

%Se expone de forma ordenada toda la tematica a tratar del curso. Debe planearse para 15 semanas.

\noindent\textbf{\textsc{Semana 1}}\\[-0.5cm]
\begin{itemize}
\item Temas: 
Presentaci\'on del curso. Unix. Consola. Comandos b\'asicos. Editores
de texto. Int\'erprete de
Python. Variables. Aritm\'etica. Listas. Diccionarios.\\[-0.6cm] 
\item Lecturas preparatorias. Videos \texttt{Introducci\'on a Unix:
  primera parte}, \texttt{Introducci\'on a Unix: segunda parte},
  \texttt{Introducci\'on a Python: primera parte}, \texttt{Python:
    listas y strings}.\\[-0.6cm] 
\item Talleres: 
\\[-0.6cm]
\end{itemize}

\noindent\textbf{\textsc{Semana 2}}\\[-0.5cm]
\begin{itemize}
\item Temas: If/while/break/continue. Ciclos. Funciones. Arreglos
(numpy). Lectura y escritura de archivos (numpy). Gr\'aficas y
visualizaci\'on (matplotlib), Objetos. \\[-0.6cm]
\item Lecturas preparatorias. Videos \texttt{Intruducci\'on a Python:
  segunda parte}, \texttt{Programaci\'on Orientada a Objetos en
  Python}, \texttt{Python: Funciones, tipos de
  variables y recursividad.}, \texttt{Numpy con IPython}, 
  \texttt{import matplotlib as plt} (los dos \'ultimos videos usan
  Ipython, no es necesario que uds. lo hagan asi.). \\[-0.6cm] 
\item Talleres: \\[-0.6cm]
\end{itemize}


\noindent\textbf{\textsc{Semana 3}}\\[-0.5cm]
\begin{itemize}
\item Temas: Integrales. Interpolaci\'on. \\[-0.6cm]
\item Lecturas preparatorias: Cap\'itulo 6 (Integration) del libro de
Landau.\\[-0.6cm]
\item Talleres: 
Publicaci\'on Taller \#1 (Unix, Python)\\[-0.6cm] 
\end{itemize}


\noindent\textbf{\textsc{Semana 4}}\\[-0.5cm]
\begin{itemize}
\item Temas: Derivadas. Ra\'ices de ecuaciones. \\[-0.6cm]
\item Lecturas preparatorias: Cap\'itulos 7.I (Numerical
  Differentiation) y 7.II (Trial-and-Error Searching) del libro de
  Landau.\\[-0.6cm] 
\item Talleres: \\[-0.6cm] 
\end{itemize}

\noindent\textbf{\textsc{Semana 5}}\\[-0.5cm]
\begin{itemize}
\item Temas: Soluci\'on de sistemas de ecuaciones lineales.\\[-0.6cm] 
\item Lecturas preparatorias:  Cap\'itulo 8 (Matrix Equation Solutions)
  del libro de Landau. \\[-0.6cm]
\item Talleres: 
Entrega soluciones Taller \#1.
Publicaci\'on Taller \#2
(Integraci\'on. Interpolaci\'on. Derivadas. Ra\'ices de ecuaciones.) 
\\[-0.6cm] 
\end{itemize}

\noindent\textbf{\textsc{Semana 6}}\\[-0.5cm]
\begin{itemize}
\item Temas: Autovalores y autovectores. Principal Component
  Analysis. \\[-0.6cm]
\item Lecturas preparatorias: Secciones 6.3.1 (Principal Component
  Regression) y 10.2 (Principal Component Analysis) del libro ISL.\\[-0.6cm]
\item Talleres: \\[-0.6cm]
\end{itemize}

\noindent\textbf{\textsc{Semana 7}}\\[-0.5cm]
\begin{itemize}
\item Temas: Transformadas de Fourier\\[-0.6cm]
\item Lecturas preparatorias: Cap\'itulo 10 (Fourier Analysis) del
  libro de Landau.\\[-0.6cm] 
\item Talleres: 
Entrega soluciones Taller \#2. 
Publicaci\'on Taller \#3
(Autovalores y
autovectores. Principal Component Analysis.)
\\[-0.6cm]
\end{itemize}

\noindent\textbf{\textsc{Semana 8}}\\[-0.5cm]
\begin{itemize}
\item Temas: Git. Github. Makefile. Unit Tests. \\[-0.6cm]
\item Videos de la serie \texttt{Version Control with Git} y
  \texttt{Automation  and Make}  en Software Carpentry.\\[-0.6cm]    
\item Talleres:\\[-0.6cm]. 
\end{itemize}

\noindent\textbf{\textsc{Semana 9}}\\[-0.5cm]
\begin{itemize}
\item Temas: C++. Introducción, sintaxis, compilar/ejecutar,
variables, ciclos. \\[-0.6cm]
\item Lecturas preparatorias: C++ Tutorial\\[-0.6cm]
\item Talleres: 
Entrega soluciones Taller \#3. 
Publicaci\'on Taller \#4 
(Sistemas de ecuaciones lineales, Transformada de Fourier. Git. Makefiles.)
\\[-0.6cm]. 
\end{itemize}


%%% SEMANA DE TRABAJO INDIVIDUAL EN 201810

\noindent\textbf{\textsc{Semana 10}}\\[-0.5cm]
\begin{itemize}
\item Temas: C++. If/while, Funciones, arreglos, Pointers, input/output \\[-0.6cm]
\item Lecturas preparatorias: C++ Tutorial\\[-0.6cm]
\item Talleres: 
\\[-0.6cm]
\end{itemize}


\noindent\textbf{\textsc{Semana 11}}\\[-0.5cm]
\begin{itemize}
\item Temas: Ecuaciones diferenciales ordinarias. \\[-0.6cm]
\item Lecturas preparatorias: Cap\'itulo 9 (ODEs) del libro de
  Landau. \\[-0.6cm] 
\item Talleres: 
\\[-0.6cm]
\end{itemize}

\noindent\textbf{\textsc{Semana 12}}\\[-0.5cm]
\begin{itemize}
\item Temas: Ecuaciones diferenciales parciales. \\[-0.6cm]
\item Lecturas preparatorias: Cap\'itulo 17 (PDEs) del libro de
  Landau. \\[-0.6cm] 
\item Talleres: 
Entrega soluciones Taller \#4. 
\\[-0.6cm]
\end{itemize}

\noindent\textbf{\textsc{Semana 13}}\\[-0.5cm]
\begin{itemize}
\item Temas: Ecuaciones diferenciales parciales. \\[-0.6cm]
\item Lecturas preparatorias: Cap\'itulo 17 (PDEs) del libro de
  Landau. \\[-0.6cm] 
\item Talleres: 
Publicaci\'on Taller \#5
(Ecuaciones Diferenciales Ordinarias, Ecuaciones Diferenciales
Parciales. Git. Makefiles.)
\\[-0.6cm]
\end{itemize}

\noindent\textbf{\textsc{Semana 14}}\\[-0.5cm]
\begin{itemize}
\item Temas: Conceptos b\'asicos de
  probabilidad y estad\'istica. Procesos aleatorios. \\[-0.6cm]  
\item Lecturas preparatorias: Cap\'itulo 1 (The Basics) del libro de
  Sivia\&Skilling. \\[-0.6cm]
\item Talleres: 
\\[-0.6cm]
\end{itemize}

\noindent\textbf{\textsc{Semana 15}}\\[-0.5cm]
\begin{itemize}
\item Temas: M\'etodos de Monte Carlo. \\[-0.6cm]
\item Lecturas preparatorias: Cap\'itulo 1 (Monte Carlo Methods) del
  libro de Krauth.\\[-0.6cm]
\item Talleres: \\[-0.6cm]
\end{itemize}

\noindent\textbf{\textsc{Primera semana de finales}}\\[-0.5cm]
\begin{itemize}
\item Talleres:
Entrega soluciones Taller \#5. 
\\[-0.6cm]
\end{itemize}


\vspace*{0.5cm} 
\stepcounter{mysection}
\noindent\textbf{\large \Roman{mysection} \quad
  Metodolog\'ia}\\[-0.2cm] 

%Se describen las tecnicas y metodos para el desarrollo exitoso del curso.

\noindent\normalsize 
Las \emph{habilidades computacionales} se desarrollan trabajando activamente. 
Por esto en las sesiones magistrales, luego de presentar un resumen de
los conceptos te\'oricos, se har\'a \'enfasis en la pr\'actica y
experimentaci\'on.   
Para que esto funcione es necesario que los estudiantes estudien el
tema correspondiente {\bf antes de cada clase} siguiendo las lecturas
preparatorias recomendadas. En el \emph{Laboratorio de M\'etodos
  Computationales} habr\'a m\'as tiempo para practicar lo visto en la
clase magistral, hacer ejercicios y aclarar dudas.   

La Magistral y el Laboratorio cuentan con repositorios  en GitHub:

\url{https://github.com/ComputoCienciasUniandes/MetodosComputacionales}. 

\url{https://github.com/ComputoCienciasUniandes/MetodosComputacionalesLaboratorio}. 


\vspace*{0.5cm} 
\stepcounter{mysection}
\noindent\textbf{\large \Roman{mysection} \quad Criterios de
  evaluaci\'on}\\[-0.2cm] 

Al comienzo del semestre se har\'a un examen (corto, sin nota) para
diagnosticar el conocimiento general que ya tienen los estudiantes
sobre los temas del curso. 

La Magistral y el Laboratorio se califican por separado. Las
componentes que reciben calificaci\'on en la Magistral son las
siguientes:

\begin{itemize}
\item 
Cinco talleres para resolver por fuera del horario de clase. 
Cada taller tiene un valor del $14\%$ de la nota definitiva.  
\item 
Ejercicios para resolver y entregar en clase. 
Se eligir\'an al azar 4 de estos ejercicios para ser calificados. 
El promedio de esas 4 notas tiene un valor del $15\%$ de la nota
definitiva. 
Esta contribuci\'on a la nota definitiva ser\'a de cero (0.0) si se
dejaron de entregar {\bf seis} o m\'as de estos ejercicios. 
\item
Un examen final (con una componente escrita y otra de
programaci\'on) con un valor del $15\%$ de la nota definitiva.  
\end{itemize}

Las componentes que reciben calificaci\'on en el Laboratorio son las
siguientes:
\begin{itemize}
\item
Ejercicios cortos para resolver  y entregar en
clase.  
Se eligir\'an al azar 6 de esos ejercicios para ser calificados. 
Cada ejercicio corresponde a un $15\%$ de la nota definitiva del
Laboratorio. 
Esta contribuci\'on a la nota definitiva ser\'a de cero (0.0) si se
dejaron de entregar {\bf tres} o m\'as de estos ejercicios. 
\item
Asistencia a clase. Cada falta a clase recibe una nota de 0.0 y cada
asistencia recibe una nota de 5.0. El promedio de estas notas
corresponde al $10\%$ de la nota definitiva.
\end{itemize}

De acuerdo a la nota definiva en el Laboratorio habr\'a {\bf un bono} en la
nota definitiva de la Magistral. 
Siendo $x$ la nota de Laboratorio, el bono correspondiente
se calcula as\'i:
$4.0 < x \leq 4.4 \rightarrow 0.1$, $4.4< x\leq 4.8\rightarrow 0.2$, $4.8<x
\leq 5.0\rightarrow 0.4$.

Todos los ex\'amenes, talleres y ejercicios ser\'an
\textbf{individuales}.  
Si en las entregas se detecta que el trabajo no fue
individual (esto incluye colaboraci\'on con personas no inscritas en
el curso), entonces la nota de todos los talleres y ejercicios quedar\'a
autom\'aticamente {\bf en cero} y se llevar\'a el caso a comit\'e
disciplinario. 

Los enunciados de los talleres se publican a m\'as tardar el lunes a
las $11$PM de la semana de publicaci\'on correspondiente. 
Las soluciones de los talleres se reciben a m\'as tardar el lunes a
las $11$PM de la semana de entrega correspondiente.

Todas las entregas de talleres y ejercicios se har\'an a trav\'es de
SICUA.  {\bf No se aceptar\'a ninguna tarea por fuera de esa
  plataforma}, a menos que ocurra un una falla en los servidores de
SICUA que afecte a {\bf todos} los estudiantes del curso.


\vspace*{0.5cm} 

\stepcounter{mysection}
\noindent\textbf{\large \Roman{mysection} \quad
  Bibliograf\'ia}\\[-0.2cm] 

%Indicar los libros y la documentacion guia.


\noindent\normalsize Bibliograf\'ia principal:

\begin{itemize}
\item
\textit{A survey of Computational Physics - Enlarged Python Book}
. R. H. Landau, M. J. P\'aez, C. C. Bordeianu. WILEY. 2012.
\url{https://psrc.aapt.org/items/detail.cfm?ID=11578}

\item
\textit{Data Analysis: A Bayesian Tutorial.} D. S. Sivia,
J. Skilling. Second Edition, Oxford Science Publications. 2006.

\item 
\textit{Statistical Mechanics: Algorithms and Computations.}
W. Krauth, Oxford Univ. Press. 

\item
\textit{An Introduction to Statistical Learning.} G. James, D. Witten,
T. Hastie, R. Tibshirani,
Springer. \url{http://www-bcf.usc.edu/~gareth/ISL/} 

\item
\textit{C++ programming for the absolute beginner.}
 M. Lee \& D. Henkemans, Second Edition, Cengage Learning, 2009.

\item
\textit{The C programming language.}
 B. Kernighan \& D. Ritchie, Second Edition, Prentice Hall.

\item Videos del curso Herramientas Computacionales que muestran los
  fundamentos de Unix y Python \url{https://www.youtube.com/playlist?list=PLHQtzvthdVM_MGC9dPFKe4hPAwBd_7RJ3}

\item Software Carpentry: \url{http://software-carpentry.org/}
\item C++ Tutorial: \url{https://www.tutorialspoint.com/cplusplus/}
\end{itemize}

\noindent\normalsize Bibliograf\'ia secundaria:
\begin{itemize}
\item
\textit{Elements of Scientific Computing}
Tveito A., Langtangen H.P., Nielsen B.F., Cai X. Spinger. 2010.


\item 
\textit{Introduction to Computation and Programming Using Python},
Guttag, J. V. The MIT Press. 2013.

\item\url{http://xkcd.com/}
\end{itemize}


\end{document}
