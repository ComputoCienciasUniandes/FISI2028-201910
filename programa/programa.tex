\documentclass[letterpaper,10pt,onecolumn]{article}
\usepackage[spanish]{babel}
\usepackage[utf8]{inputenc}
\usepackage{amsfonts}
\usepackage{amsthm}
\usepackage{amsmath}
\usepackage{mathrsfs}

\usepackage{enumitem}
\usepackage[pdftex]{color,graphicx}
\usepackage{hyperref}
\usepackage{listings}
\usepackage{calligra}
\usepackage{url}
%\usepackage{algpseudocode} 
\DeclareMathAlphabet{\mathcalligra}{T1}{calligra}{m}{n}
\DeclareFontShape{T1}{calligra}{m}{n}{<->s*[2.2]callig15}{}
\newcommand{\scripty}[1]{\ensuremath{\mathcalligra{#1}}}
\lstloadlanguages{[5.2]Mathematica}
\setlength{\oddsidemargin}{0cm}
\setlength{\textwidth}{490pt}
\setlength{\topmargin}{-40pt}
\addtolength{\hoffset}{-0.3cm}
\addtolength{\textheight}{4cm}

\begin{document}
\begin{center}

\includegraphics[width=490pt]{header.png}\\[0.5cm]

\textsc{\LARGE M\'etodos Computacionales}\\[0.1cm]
%\large Jaime E. Forero Romero\\[0.5cm]

\end{center}

\large \noindent\textsc{Nombre del curso:}  M\'etodos Computacionales%Aqui  
                                %nombre del curso 
  
\noindent\textsc{C\'odigo del curso:} FISI 2028 (Magistral) FISI 2029 (Laboratorio) %Aqui el codigo del
                                %curso 

\noindent\textsc{Unidad acad\'emica:} Departamento de F\'isica

\noindent\textsc{Periodo acad\'emico:} 201910 %Aqui el periodo,
                                %p.ej. 201510 

\noindent\textsc{Horario (magistral - Secci\'on 1):} Ma y Ju, 6:30 a 7:50.

\noindent\textsc{Horario (magistral - Secci\'on 2):} Ma y Ju, 12:30 a 13:50.
%Aqui el horario, %p.ej. Ma y Ju, 10:00 a 11:20 

\noindent\textsc{Horario (laboratorio - Secci\'on 1):} Ma 15:30 - 16:50

\noindent\textsc{Horario (laboratorio - Secci\'on 2):} Mie 14:00 - 15:20

\noindent\rule{\textwidth}{1pt}\\[-0.3cm]

\normalsize \noindent\textsc{Nombre profesor magistral (Secci\'on 1):}
Jaime Ernesto Forero Romero%Aqui nombre del profesor principal   

\noindent\textsc{Correo electr\'onico:}
\href{mailto:je.forero@uniandes.edu.co}{\nolinkurl{je.forero@uniandes.edu.co}}
%Cambie address por su direccion de correo uniandes 

\noindent\textsc{Horario y lugar de atenci\'on:} Martes 16:00 a 17:00, Jueves 16:00 a 17:00, Oficina Ip208 
\\[-0.1cm]

\normalsize \noindent\textsc{Nombre profesora magistral (Secci\'on 2):}
Veronica Arias Callejas %Aqui nombre del profesor principal 

\noindent\textsc{Correo electr\'onico:}
\href{mailto:v.arias@uniandes.edu.co}{\nolinkurl{v.arias@uniandes.edu.co}}
%Cambie address por su direccion de correo uniandes 

\noindent\textsc{Horario de atenci\'on:} con cita previa. 
\\[-0.1cm]


\normalsize \noindent\textsc{Nombre profesor Laboratorio:}
Jos\'e Alejandro Monta\~na Cort\'es %Aqui nombre del profesor principal 

\noindent\textsc{Correo electr\'onico:}
\href{mailto:ja.montana@uniandes.edu.co}{\nolinkurl{ja.montana@uniandes.edu.co}}
%Cambie address por su direccion de correo uniandes 

\noindent\textsc{Horario de atenci\'on:} con cita previa. 
\\[-0.1cm]
%\href{mailto: jd.prada1760@uniandes.edu.co}{\nolinkurl{jd.prada1760@uniandes.edu.co}}

%Cambie address por direccion de correo uniandes del profesor
%complementario 

%\noindent\textsc{Horario y lugar de atenci\'on:} %Aqui horario y
%lugar de atencion del profesor complementario, p.ej. Vi, 15:00 a
%17:00, Oficina Ip102 
%\\[-0.1cm]
%Repetir esto en caso de varios profesores complementarios

%\noindent\textsc{Nombre monitor(a):} %Aqui nombre del monitor si aplica

%\noindent\textsc{Correo electr\'onico:}
%\href{mailto:address@uniandes.edu.co}{\nolinkurl{address@uniandes.edu.co}}
%%Cambie address por direccion de correo uniandes del monitor 

%\noindent\textsc{Horario y lugar de atenci\'on:} %Aqui horario y
%lugar de atencion del monitor, p.ej. Vi, 15:00 a 17:00, Oficina Ip102 

\noindent\rule{\textwidth}{1pt}\\[-0.1cm]

\newcounter{mysection}
\addtocounter{mysection}{1}

\noindent\textbf{\large \Roman{mysection} \quad Introducci\'on}\\[-0.2cm]

%Este espacio es para hacer una introduccion al curso, evidenciando la
%propuesta metodologica. Debe ser clara y precisa. 

\noindent\normalsize Los m\'etodos computacionales son un aspecto
inseparable de cualquier \'area de trabajo en ciencia e ingenier\'ia.
Esto se debe a la facilidad de acceso a 
computadoras programables  y al aumento exponencial en su capacidad de
procesamiento. 
Este curso busca guiar a los estudiantes en el desarrollo de sus
\emph{habilidades computacionales} parar obtener informaci\'on sobre
la realidad que los datos generados por computador pretenden
describir.    
Estos datos pueden ser mediciones o simulaciones de
sistemas f\'isicos, qu\'imicos, geol\'ogicos, biol\'ogicos,
financieros o industriales, entre otros.     
El programa del curso tiene dos componentes diferenciadas: m\'etodos
num\'ericos  y \emph{carpinter\'ia} de software.  
La parte de m\'etodos num\'ericos busca ilustrar el paso entre la
formulaci\'on matem\'atica de una pregunta sobre la realidad y su
descripci\'on num\'erica/computacional.
El objetivo principal es mostrar diferentes formas de usar un
computador para formular esa pregunta.
La parte de carpinter\'ia presenta algunas t\'ecnicas y
pr\'acticas necesarias, no suficientes, para poder obtener resultados
computacionales reproducibles.     
\\[0.1cm] 

\stepcounter{mysection}
\noindent\textbf{\large \Roman{mysection} \quad Objetivos}\\[-0.2cm]

%En este espacio se debe precisar el ente visor del curso y el
%proposito ideal al finalizar el curso. 
\noindent\normalsize El objetivo principal del curso es presentar
algoritmos y t\'ecnicas b\'asicas para:

\begin{itemize}
\item resolver ecuaciones diferenciales ordinarias y parciales, \\[-0.6cm]
\item analizar y describir datos con t\'ecnicas estad\'isticas basadas
  en m\'etodos Monte Carlo, \\[-0.6cm]
\item desarrollar esquemas reproducibles para el an\'alisis de datos cient\'ificos. \\[-0.6cm]
\end{itemize} 
\vspace*{0.5cm} 

\stepcounter{mysection}
\noindent\textbf{\large \Roman{mysection} \quad Competencias a
  desarrollar}\\[-0.2cm] 

%En este espacio se describen las habilidades que el estudiante desarrollara en el transcurso del curso.

\noindent\normalsize Al finalizar el curso, se espera que el
estudiante est\'e en capacidad de: 

\begin{itemize}
\item implementar algoritmos sencillos para la resoluci\'on de
  ecuaciones diferenciales y para el an\'alisis estad\'istico
  exploratorio de datos, \\[-0.6cm]   
\item manejar la sintaxis b\'asica de dos lenguajes de programaci\'on
  modernos de computaci\'on num\'erica: uno de alto nivel y otro de
  bajo nivel,\\[-0.6cm]  
\item desarrollar un esquema sencillo para obtener resultados
  computacionales reproducibles.\\[-0.6cm]  
\end{itemize}

\vspace*{0.5cm} 

\stepcounter{mysection}
\noindent\textbf{\large \Roman{mysection} \quad Contenido por
  semanas}\\[-0.2cm]  

%Se expone de forma ordenada toda la tematica a tratar del curso. Debe planearse para 15 semanas.

\noindent\textbf{\textsc{Semana 1.}} Presentaci\'on del curso. Binder. Repaso Unix. Repaso Python. \\[-0.5cm]
\begin{itemize}
\item Temas: 
Presentaci\'on del curso. Binder. Comandos b\'asicos Unix. 
Sintaxis b\'asica de Python. \\[-0.6cm] 
\item Lecturas preparatorias: Los siguientes videos de Herramientas Computacionales

\begin{itemize}
\item \texttt{Introducci\'on a Unix:
  primera parte}
\item \texttt{Introducci\'on a Unix: segunda parte}
\item \texttt{Introducci\'on a Python: primera parte}
\item \texttt{Python: listas y strings}.
\item  \texttt{Intruducci\'on a Python: segunda parte}
\end{itemize}
\end{itemize}


\noindent\textbf{\textsc{Semana 2.}} Repaso Python. \\[-0.5cm]
\begin{itemize}
\item Temas: 
Funciones de Python. Numpy y Matplotlib.\\[-0.6cm] 
\item Lecturas preparatorias: Los siguientes videos de Herramientas Computacionales

\begin{itemize}
\item \texttt{Python: Funciones, tipos de variables y recursividad.}
\item \texttt{Numpy con IPython}
\item \texttt{import matplotlib as plt}
\end{itemize}
\end{itemize}

\noindent\textbf{\textsc{Semana 3}}\\[-0.5cm]
\begin{itemize}
\item Temas: Conceptos b\'asicos de
  probabilidad y estad\'istica. Procesos aleatorios. \\[-0.6cm]  
\item Lecturas preparatorias: 
Cap\'itulo 5 (Monte Carlo Simulations) del libro de Landau.\\[-0.6cm]
\end{itemize}

\noindent\textbf{\textsc{Semana 4}}\\[-0.5cm]
\begin{itemize}
\item Temas: M\'etodos de Monte Carlo y Estimaci\'on bayesiana de par\'ametros. \\[-0.6cm]
\item Lecturas preparatorias:  Cap\'itulo 1 (The Basics) del libro de
  Sivia\&Skilling.  Cap\'itulo 1 (Monte Carlo Methods) del
  libro de Krauth.\\[-0.6cm]
\end{itemize}

\noindent\textbf{\textsc{Semana 5}}\\[-0.5cm]
\begin{itemize}
\item Temas: Integraci\'on. \\[-0.6cm]
\item Lecturas preparatorias: Cap\'itulo 6 (Integration) del libro de
Landau.\\[-0.6cm]
\end{itemize}

\noindent\textbf{\textsc{Semana 6}}\\[-0.5cm]
\begin{itemize}
\item Temas: Derivadas. Ra\'ices de ecuaciones. \\[-0.6cm]
\item Lecturas preparatorias: Cap\'itulos 7.I (Numerical
  Differentiation) y 7.II (Trial-and-Error Searching) del libro de
  Landau.\\[-0.6cm] 
\end{itemize}

\noindent\textbf{\textsc{Semana 7}}\\[-0.5cm]
\begin{itemize}
\item Temas: Soluci\'on de sistemas de ecuaciones lineales. Ajustes
  por m\'inimos cuadrados. \\[-0.6cm] 
\item Lecturas preparatorias:  Cap\'itulo 8 (Matrix Equation Solutions)
  del libro de Landau. \\[-0.6cm]
\item Talleres: Publicaci\'on Taller \#1. \\[-0.6cm]
\end{itemize}


\noindent\textbf{\textsc{Semana 8}}\\[-0.5cm]
\begin{itemize}
\item Temas: Autovalores y autovectores. 
An\'alisis de Componentes Principales. \\[-0.6cm]
\item Lecturas preparatorias: Secciones 6.3.1 (Principal Component
  Regression) y 10.2 (Principal Component Analysis) del libro ISL.\\[-0.6cm]
\item Talleres: Entrega Taller \#1. \\[-0.6cm]
\end{itemize}

\noindent\textbf{\textsc{Semana 9}}\\[-0.5cm]
\begin{itemize}
\item Temas: Transformadas de Fourier. Se\~nales y Filtros. \\[-0.6cm]
\item Lecturas preparatorias: Cap\'itulo 10 (Fourier Analysis) del
  libro de Landau.\\[-0.6cm] 
\end{itemize}

\noindent\textbf{\textsc{Semana 10}}\\[-0.5cm]
\begin{itemize}
\item Temas: Git. Github. Unit Tests. \\[-0.6cm]
\item Videos de la serie \texttt{Version Control with Git} y
  \texttt{Python Testing}  en Software Carpentry.\\[-0.6cm]    
\end{itemize}

\noindent\textbf{\textsc{Semana 11}}\\[-0.5cm]
\begin{itemize}
\item Temas: C++. Introducción, sintaxis, compilar/ejecutar,
variables, ciclos. \\[-0.6cm]
\item Lecturas preparatorias: C++ Tutorial\\[-0.6cm]
\end{itemize}


\noindent\textbf{\textsc{Semana 12}}\\[-0.5cm]
\begin{itemize}
\item Temas: C++. If/while, Funciones, arreglos, Pointers, input/output. Makefiles \\[-0.6cm]
\item Lecturas preparatorias: C++ Tutorial. Videos de Sofware
  Carpentry sobre Makefiles.\\[-0.6cm]  
\end{itemize}

\noindent\textbf{\textsc{Semana 13}} (Semana de trabajo individual)

\noindent\textbf{\textsc{Semana 14}}\\[-0.5cm]
\begin{itemize}
\item Temas: Ecuaciones diferenciales ordinarias. \\[-0.6cm]
\item Lecturas preparatorias: Cap\'itulo 9 (ODEs) del libro de
  Landau. \\[-0.6cm] 
\end{itemize}

\noindent\textbf{\textsc{Semana 15}}\\[-0.5cm]
\begin{itemize}
\item Temas: Ecuaciones diferenciales parciales (Ecuaci\'on de
  Poisson. Ecuaci\'on de Difusi\'on). \\[-0.6cm]
\item Lecturas preparatorias: Cap\'itulo 17 (PDEs) del libro de
  Landau. \\[-0.6cm] 
\item Talleres: Publicaci\'on Taller \#2. \\[-0.6cm]
\end{itemize}


\noindent\textbf{\textsc{Semana 16}}\\[-0.5cm]
\begin{itemize}
\item Temas: Ecuaciones diferenciales parciales (Ecuaci\'on de Onda). \\[-0.6cm]
\item Lecturas preparatorias: Cap\'itulo 18 (PDEs) del libro de
  Landau. \\[-0.6cm] 
\item Talleres: Entrega Taller \#2. \\[-0.6cm]
\\[-0.6cm]
\end{itemize}




\vspace*{0.5cm} 
\stepcounter{mysection}
\noindent\textbf{\large \Roman{mysection} \quad
  Metodolog\'ia}\\[-0.2cm] 

%Se describen las tecnicas y metodos para el desarrollo exitoso del curso.

\noindent\normalsize 
Las \emph{habilidades computacionales} se desarrollan trabajando activamente. 
Por esto en las sesiones magistrales, luego de presentar un resumen de
los conceptos te\'oricos, se har\'a \'enfasis en la pr\'actica y
experimentaci\'on.   
Para que esto funcione es necesario que los estudiantes estudien el
tema correspondiente {\bf antes de cada clase} siguiendo las lecturas
preparatorias recomendadas. En el \emph{Laboratorio de M\'etodos
  Computationales} habr\'a m\'as tiempo para practicar lo visto en la
clase magistral, hacer ejercicios y aclarar dudas.   

La Magistral y el Laboratorio cuentan con repositorios  en GitHub:

\url{https://github.com/ComputoCienciasUniandes/FISI2028-201910}. 

\url{https://github.com/ComputoCienciasUniandes/FISI2029-201910}. 


\vspace*{0.5cm} 
\stepcounter{mysection}
\noindent\textbf{\large \Roman{mysection} \quad Criterios de
  evaluaci\'on}\\[-0.2cm] 

La Magistral y el Laboratorio se califican por separado. 
Las componentes que reciben calificaci\'on en la Magistral (en
par\'entesis su contribuci\'on a la nota definitiva) son las
siguientes:  

\begin{itemize}
\item Asistencia ($15\%$). Cada asistencia a clase cuenta como una
  nota de 5.0 y una falta como 0.0. El promedio de esas notas ser\'a
  la nota de asistencia. 
  Si hay {\bf seis} o m\'as fallas no justificadas durante todo el
  semestre esta nota es cero (0.0).
\item Ejercicios ($30\%$). En cada clase hay un ejercicio para
  entregar. Cada ejercicio tiene dos partes. 
  La primera se publica al menos un d\'ia y debe resolverse antes de
  llegar a clase.
  La segunda se publica y se resuelve durante la clase.
  Durante el semestre el profesor eligir\'a a su discreci\'on seis (6)
  de estos ejercicios para ser calificados. 
  El promedio de esas notas ser\'a la nota de ejercicios.
  Si se dejaron de entregar {\bf seis} o m\'as ejercicios (sin
  justificaci\'on) durante todo el semestre esta nota es
  (0.0). 
\item Talleres ($30\%$). Habr\'a dos talleres para resolver por fuera de
  clase. El promedio de esas calificaciones ser\'a la nota de talleres.
\item Examen Final ($25\%$). Tendr\'a una componente escrita y otra de
  programaci\'on. 
\end{itemize}

Las componentes que reciben calificaci\'on en el Laboratorio son las
siguientes:
\begin{itemize}
\item Asistencia ($30\%$) Similar a la Magistral.
  Si hay {\bf tres} o m\'as fallas no justificadas durante todo el
  semestre esta nota es cero (0.0).
\item Ejercicios ($70\%$). En cada clase hay un ejercicio para
  entregar. Se desarrollan y entregan en clase.
  Durante el semestre el profesor eligir\'a a su discreci\'on cinco
  (5) de estos ejercicios para ser calificados.
  El promedio de esas notas ser\'a la nota de ejercicios.
  Si se dejaron de entregar {\bf tres} o m\'as ejercicios (sin
  justificaci\'on) durante todo el semestre esta nota es
  cero (0.0). 
\end{itemize}

De acuerdo a la nota definiva en el Laboratorio habr\'a {\bf un bono} en la
nota definitiva de la Magistral. 
Siendo $x$ la nota de Laboratorio, el bono $b$  se calcula as\'i:
$4.0 < x \leq 4.4 \rightarrow b=0.1$; $4.4< x\leq 4.8\rightarrow b=0.2$; $4.8<x
\leq 5.0\rightarrow b=0.3$.

Todos los ex\'amenes, talleres y ejercicios son
\textbf{individuales}.  
Si en las entregas se detecta que el trabajo no fue
individual (esto incluye colaboraci\'on con personas no inscritas en
el curso, i.e. a trav\'es de ``monitor\'ias'') se llevar\'a el caso a
comit\'e disciplinario y la nota del curso queda como Pendiente
Disciplinario hasta que el comit\'e tome alguna decisi\'on. 

Todas las entregas de talleres y ejercicios se har\'an a trav\'es de
SICUA.  {\bf No se aceptar\'a ninguna tarea por fuera de esa
  plataforma}, a menos que ocurra un una falla en los servidores de
SICUA que afecte a {\bf todos} los estudiantes del curso.


\vspace*{0.5cm} 

\stepcounter{mysection}
\noindent\textbf{\large \Roman{mysection} \quad
  Bibliograf\'ia}\\[-0.2cm] 

%Indicar los libros y la documentacion guia.


\noindent\normalsize Bibliograf\'ia principal:

\begin{itemize}
\item
\textit{Software Carpentry: Python Testing}
\url{http://katyhuff.github.io/python-testing/}

\item
\textit{A survey of Computational Physics - Enlarged Python Book}
. R. H. Landau, M. J. P\'aez, C. C. Bordeianu. WILEY. 2012.
\url{https://psrc.aapt.org/items/detail.cfm?ID=11578}

\item
\textit{Data Analysis: A Bayesian Tutorial.} D. S. Sivia,
J. Skilling. Second Edition, Oxford Science Publications. 2006.

\item 
\textit{Statistical Mechanics: Algorithms and Computations.}
W. Krauth, Oxford Univ. Press. 

\item
\textit{An Introduction to Statistical Learning.} G. James, D. Witten,
T. Hastie, R. Tibshirani,
Springer. \url{http://www-bcf.usc.edu/~gareth/ISL/} 

\item
\textit{C++ programming for the absolute beginner.}
 M. Lee \& D. Henkemans, Second Edition, Cengage Learning, 2009.

\item
\textit{The C programming language.}
 B. Kernighan \& D. Ritchie, Second Edition, Prentice Hall.

\item Videos del curso Herramientas Computacionales que muestran los
  fundamentos de Unix y Python \url{https://www.youtube.com/playlist?list=PLHQtzvthdVM_MGC9dPFKe4hPAwBd_7RJ3}

\item Software Carpentry: \url{http://software-carpentry.org/}
\item C++ Tutorial: \url{https://www.tutorialspoint.com/cplusplus/}
\end{itemize}

\noindent\normalsize Bibliograf\'ia secundaria:
\begin{itemize}
\item
\textit{Elements of Scientific Computing}
Tveito A., Langtangen H.P., Nielsen B.F., Cai X. Spinger. 2010.


\item 
\textit{Introduction to Computation and Programming Using Python},
Guttag, J. V. The MIT Press. 2013.

\item\url{http://xkcd.com/}
\end{itemize}


\end{document}
